
In this study, it was demonstrated the capabilities of a deep learning approach for the automatic detection and identification of small boats in the waters surrounding three cities of the Gulf of California  with a precision of up to 74\%. This work used CNNs to identify types of small vessels. Specifically, this work presented an image detection model capable of distinguishing small boats in the Gulf of California with an accuracy of up to 93.9\%, and encouraging result considering the high variability of the input images.

The research done has shown that through deep learning techniques it is possible to detect small boats and categorizing them even with large and highly ambiguous datasets. It was seen from the different tests done with the imagery dataset that image sharping improves the model accuracy which implies that access to better quality imagery, such as the ones available through pay services, could improve considerably the model precision and training times.


Our results have several important implications. First, the study used satellite data to predict the number and types of ships in three important cities located in the Gulf of California. The resulting analysis can contribute to the region's shipping carbon inventory by adding the emissions produced by the small boat fleet. Further, through this approach, it is also possible to assign emissions into regions that can support the development of policies that can mitigate local GHG and air pollution. In addition, the transfer learning algorithm can be pre-trained in advance and immediately applied to any sea area in the world. This will provide a potential way to increase efficiency for scientists and engineers around the world who need to estimate local maritime emissions. In addition, the model can quickly and accurately identify the length of the boat, allowing researchers to allocate more time to the vessels they need to know about, not just small boats. Finally, all of the above benefits can be exploited in underserved areas where there is a shortage of researchers.

Our study has some limitations. This work considered the ship as a single detection object, and did not evaluate whether the model can improve the accuracy of identifying ships in the case of multiple detection objects. By downsampling the image to a 416 pixels × 416 pixels, it is possible to mask some of the boats that are at the edges of the picture. Furthermore, a considerable proportion of the images in our dataset are very high-definition. This is markedly different from what we need to identify with the Gulf of California, and may affect the final test results. Although the detection task in this case is more challenging, the model still achieves excellent performance in the classification of normal studies and is well suited for recognizing small boats. We note that due to the low data quality of the detected regions, they are less suitable as training datasets. However, using datasets from other regions or higher quality open-source imagery may result in an inaccurate coverage of all types of ships in the region. When looking for the classification category and the data under this category, the subject of ensuring that the environment in which the object is located is the same as the environment in which the object in other categories is located through the AI fairness principle deserves further study. From this point of view, large-scale collection of data sources in the real physical world would be costly. Further work on the recognition process could be achieved by using reinforcement learning to generate new data sources for training data, or building simulations in the virtual world that can be used as training data environment.

However, the model only detects and classifies certain types of small boats. In order to have estimations of fuel consumption and emissions, it is necessary to couple the BoatNet model with small boat behavior datasets~\cite{ferrer2021mexican}, typical machinery and fuel characteristics, and emission factors~\cite{inecc2020inventario}. 

Finally, this work have demonstrated that deep learning models have the potential to identify small boats in extreme environments at performance levels that provide practical value. With further analysis and small boat data sources, these methods may eventually allow for a rapid assessment of shipping carbon inventories.