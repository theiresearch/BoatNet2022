
In this study, I introduce the reader to satellite image recognition and how I can make a modest contribution to the development of the field.

In Chapter~\ref{chap:2}, I review the traditional research literature on estimating the number of small boats. Researchers use kernel density estimation to distribute population and vessel numbers and ultimately show that their predictions can accurately predict the number of vessels in the Gulf fisheries. Some scholars then focused on describing the combination of national small vessel registries and surveyed vessel discharge results. However, the methodology does not cover states without a national small boat registry, and small boats are not just recreational vessels.

Obviously, these traditional approaches do not extract information well from a large amount of data. Although as early as the 1990s, researchers such as Yann LeCun realized that convolutional neural networks could recognize images. However, with the exponential growth in computing power in the last five years, scientists and engineers have only been able to apply convolutional neural networks to topics such as large-scale image recognition and object detection. Since 2012, the field has been reinvented by creating large-scale supervised datasets and developing neural object detection network models. Although it has only been nine years since then, the field has evolved at an amazing rate. Innovations in building better datasets and more effective models have alternated, and both have contributed to the field's growth.

In Chapter~\ref{chap:3}, I first explain how convolutional neural networks can detect a letter `X' and describe the mathematical principles involved. Next, I introduce the two most commonly used neural network frameworks today: Faster RCNN and YOLO and cite Dwivedi's work to illustrate to the reader the advantages of the YOLO model in recognizing small objects advantage of higher accuracy in recognizing objects in videos. Next, I show the data needed in training the model and the preparation needed to train the model faster. Finally, to speed up the model's training, I use a GPU to train the model and use Google Colab, Google Drive and GitHub as my testing and development tools.


In the subsequent section, I highlight the inequality in the amount of data available on satellite images for the Gulf of California. Thus, I took the approach of selecting one of the cities with a large amount of satellite data every year and then analyzing it. However, even this does not avoid the fact that the quality of the satellite images in 2019 is worse compared to the following two years. In response, I took the approach of sharpening the images to bring out the details of the images. As a result, the model can improve the object recognition rate by around 26\%.

Then, I designed algorithms to detect the length of small boats. Since I did not have access to the zoom scale of Google Earth Pro, I had to define a zoom scale myself to find the relationship between the length of the real boat and the boat in the image. Finally, by using the same eye altitude and resolution for all the photos in the test dataset, the algorithm can automatically detect the boat's length. In the test, the boat's length detected by the algorithm is similar to the length measured from Google Earth Pro.

In Chapter~\ref{chap:4}, I show the detection results that can combine open high-resolution satellite data and convolutional neural networks. The results are not as high as the ideal recognition rate, but such detection results are still acceptable due to the inferior quality of the monitoring data. 

The results show a divergence. In 2019, there were on average 100.63 small boats in Guaymas. Although this value dropped to 91.55 in 2020, it quickly rises again to 147.75 in 2021. Similarly, in 2019, 2020 and 2021, there are 31, 46.55 and 39 small boats in Santa Rosalia, respectively. However, Loreto did not show such an upward trend in general. In 2019, 2020 and 2021, there are 37.13, 29.22 and 27.33 small boats in Loreto, respectively. On the one hand, there is a large port like Guaymas with many small ships. On the other hand, small ports like Loreto and Santa Rosalia have almost no big ships and fewer small ships. But it is interesting to note that small boats in Guaymas, Loreto and Santa Rosalia are almost certainly small boats for family use for recreation. The analysis of satellite images shows that most small boats are docked in the harbour, which means it is rare to see a large number of small boats floating on the sea. Another interesting point is that the number of small boats identified increases as the year increases. Considering that most of the boats are in the harbour and that the models used for the tests are identical, the only difference is that the detail of the satellite images is better realizable each year, i.e., each year we have an image with higher quality. Then, it is reasonable to believe that the increase in the quality of satellite photos can improve the quality of object detection.



% 4. Include a brief reflection on what I have learnt from undertaking my project as far as project management is concerned.
 
%\section{Brief Reflection}
All in all, I am very excited about the progress I have made in this area over the past four months and are pleased to contribute to the field. At the same time, I am convinced that there is still a long way to get AI to the point where it can detect objects beyond the human level, and we still face huge challenges and many outstanding questions that need to be addressed in the future. One key challenge is that we still do not have a good way to handle deeper object detection ---- those problems that require understanding photos or video inference ---- for example, the question of whether the two boats are one boat. In the future, we will also have to address the complex problem of integrating object detection and natural language processing to reach a level that allows AI to understand video.

We also hope to encourage more researchers to work on applying satellite image detection to new areas. We believe it will lead us to build better agents that can understand media and hope to see these ideas implemented and developed in industry applications.